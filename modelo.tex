\NeedsTeXFormat{LaTeX2e}
%-----------------------------------------------------------
\documentclass[a4paper,12pt]{monografia}
\usepackage[portuguese, colorinlistoftodos, textsize=tiny]{todonotes}
\usepackage{amsmath,amsthm,amsfonts,amssymb}
\usepackage[mathcal]{eucal}
\usepackage{latexsym}
\usepackage[portuguese]{babel}  
\usepackage[utf8]{inputenc}
\usepackage{setspace}
%\usepackage{natbib}
\usepackage{bm}
\usepackage[portuguese,algoruled,longend, linesnumbered]{algorithm2e}
\usepackage{listings}
\usepackage{graphicx}
\usepackage{hyperref}
\hypersetup{colorlinks,
   debug=false,
   linkcolor=black,  %%% cor do tableofcontents, \ref, \footnote, etc
   citecolor=black,  %%% cor do \cite
   urlcolor=black,   %%% cor do \url e \href
   bookmarksopen=true,
}
\usepackage[alf,bibjustif]{abntex2cite}

\newcounter{todocounter}
\newcommand{\comment}[2][]
{\stepcounter{todocounter}\todo[caption={\thetodocounter: #2}, #1] 
{\begin{spacing}{1}\thetodocounter: #2\end{spacing}}}
\reversemarginpar
\setlength{\marginparwidth}{2.5cm}
\lstloadlanguages{C}
%-----------------------------------------------------------
%-----------------------------------------------------------
\theoremstyle{plain}
\newtheorem{theorem}{Teorema}[section]
\newtheorem{axiom}{Axioma}[section]
\newtheorem{corollary}{Corolário}[section]
\newtheorem{lemma}{Lema}[section]
\newtheorem{proposition}{Proposição}[section]
%-----------------------------------------------------------
\theoremstyle{definition}
\newtheorem{definition}{Definição}[section]
\newtheorem{example}{Exemplo}[section]
%-----------------------------------------------------------
\theoremstyle{remark}
\newtheorem{remark}{Observação}[section]
%-----------------------------------------------------------
%-----------------------------------------------------------
\newcommand{\R}{\mathbb{R}}
\newcommand{\N}{\mathbb{N}}
\newcommand{\Z}{\mathbb{Z}}
\newcommand{\Q}{\mathbb{Q}}
\newcommand{\K}{\mathbb{K}}
\newcommand{\I}{\mathbb{I}}
\newcommand{\id}{\mathbf{1}}
\newcommand{\U}{\mathcal{U}}
\newcommand{\V}{{\cal V}}
%-----------------------------------------------------------
\def\ind{\hbox{ ind }}
%-----------------------------------------------------------
\include{hifenizacao}



\begin{document}
%%%%%%%%%%%%%%%%%%%%%%%%%%%%%%%%%%%%%%%%%%%%%%%%%%%%%%%%%%%5
%
%                  INFORMAÇÕES PRÉ-TEXTUAIS
%
%----------------- Título e Dados do Autor -----------------
\titulo{$<<$Título da monografia$>>$}
\subtitulo{$<<$Subtítulo - opcional$>>$} % opcional
\autor{$<<$Nome do aluno$>>$} \nome{$<<$Primeiro Nome$>>$} \ultimonome{$<<$\'Ultimo Nome$>>$}
%
%---------- Informe o Curso e Grau -----
\mestrado %Pode ser \bacharelado \licenciatura \especializacao \mestrado ou \doutorado
\curso{$<<$Nome do curso$>>$} 
\dia{$<<$dia$>>$} \mes{$<<$m\^es$>>$} \ano{$<<$ano$>>$} % data da aprovação
\cidade{$<<$Juiz de Fora$>>$}
%
%----------Informações sobre a Instituição -----------------
\instituicao{$<<$Nome da universidade$>>$} \sigla{$<<$UFJF$>>$}
\unidadeacademica{$<<$Nome do instituto$>>$}
\departamento{$<<$Nome do departamento$>>$}
%
%------Nomes do Orientador, 1o. Examinador e 2o. Examinador-
\orientador{$<<$Nome do Orientador$>>$}
\ttorientador{$<<$Título do Orientador$>>$}
%
\coorientador{$<<$Nome do Co-orientador$>>$} % opcional
\ttcoorientador{$<<$Título do Co-orientador$>>$} % se digitado \coorientador
%
\examinadorum{$<<$Nome do Examinador 1$>>$}
\ttexaminadorum{$<<$Título do Examinador 1$>>$}
%
\examinadordois{$<<$Nome do Examinador 2$>>$}
\ttexaminadordois{$<<$Título do Examinador 2$>>$}
%
\examinadortres{Nome do Examinador 3}
\ttexaminadortres{Título do Examinador 3}
%
\examinadorquatro{Nome do Examinador 4}
\ttexaminadorquatro{Título do Examinador 4}
%
%-------- Informações obtidas na Biblioteca ----------------
%
%\CDU{536.21} \areas{1.Análise Matemática  2. Topologia.}
%\npaginas{xx}  % total de páginas do trabalho
%
%%%%%%%%%%%%%%%%%%%%%%%%%%%%%%%%%%%%%%%%%%%%%%%%%%%%%%%%
%    DADOS DE FORMATACAO DA MONOGRAFIA
%
% A instrução abaixo insere a logo do curso e da instituicao na capa da monografia. Basta comentar caso não queira os logos. Para alterar o logo da instituicao e curso, basta alterar os arquivos logoInstituicao.png e logoCurso.jpg. Caso deseje alterar os arquivos, os substituta por imagens do mesmo tamanho!
%\inserirlogo  
%
%
%
%          FIM DAS INFORMAÇÕES PRÉ-TEXTUAIS
%
%%%%%%%%%%%%%%%%%%%%%%%%%%%%%%%%%%%%%%%%%%%%%%%%%%%%%%%%%

\maketitle









%----------------------------dedicatória  opcional--------------
\begin{dedicatoria}
Aos meus amigos e irmãos.\\
Aos pais, pelo apoio e sustento.\\
\end{dedicatoria}


%--------Digite aqui o seu resumo em Português--------------
\resumo{Resumo} As instruções aqui contidas objetivam auxiliar os autores na preparação de documentos para impressão de monografias do Departamento de Ciência da Computação. Os estilos encontram-se definidos em um modelo denominado Monografia.cls. O resumo deve ser escrito na mesma língua do texto (Português, Inglês ou Espanhol) e descreve o conteúdo do texto em cerca de 150-200 palavras. Esta é a primeira versão das instruções e dos formatos e, portanto, sujeita a incorreções e omissões. Sugestões de melhorias são muito benvindas: envie mensagem para jairo.souza@ufjf.edu.br.

\noindent \\ \textbf{Palavras-chave:} Monografia, latex, instruções.



%-----------Digite aqui o seu resumo em Inglês--------------
\resumo{Abstract} You must summarize your work in 150-200 words.

\noindent \\ \textbf{Keywords:} Monograph, latex, instructions.


%-----------Ou digite aqui o seu resumo em Frances----------
%\resumo{Resumo} C'est un modúle de la monographie dans \LaTeX et
%5utilise la classe monografia.cls, avec le but de aider dans le
%maniement des travaux de conclusion des plusieurs cours de
%l'Université Fédérale de Juiz de Fora.



%-----------------------------------------------------------
\agradecimento{Agradecimentos} \indent\indent 
A todos os meus parentes, pelo encorajamento e
apoio.

Ao professor Beltrano pela orientação, amizade e
principalmente, pela paciência, sem a qual este trabalho não se
realizaria.


Aos professores do Departamento de Ciência da Computação pelos seus
ensinamentos e aos funcionários do curso, que durante esses anos,
contribuíram de algum modo para o nosso enriquecimento pessoal e
profissional.
\newpage


%---------------------- EPÍGRAFE I (OPCIONAL)--------------
\begin{epigrafe}
``Lembra que o sono é sagrado e alimenta de horizontes o tempo acordado de viver''.\\
\hfill Beto Guedes (Amor de Índio)
\end{epigrafe}



%----Sumário, lista de figura e de tabela ------------
 \tableofcontents \thispagestyle{empty} \listoffigures
\thispagestyle{empty} \listoftables \thispagestyle{empty}



%----Glossário ------------
\chapter*{Lista de Abreviações} \addcontentsline{toc}{chapter}{Lista de Abreviações}
\doublespacing  \begin{tabular}{l l}

DCC & Departamento de Ciência da Computução \\
UFJF & Universidade Federal de Juiz de Fora \\



\end{tabular}  \thispagestyle{empty}
%---------------------




%%%%%%%%%%%%%%%%%%%%%%%%%%%%%%%%%%%%%%%%%%%%%%%%%%%%%%%%%%%
%
%--------------Início do Conteúdo---------------------------
%
%
\pagestyle{ruledheader}
\chapter{Introdução}
Este modelo pretende atender às necessidades de padronização dos trabalhos de monografia do Departamento de Ciência da Computação, Instituto de Ciências Exatas da Universidade Federal de Juiz de Fora e servir de guia para alunos e professores.
 
Elaborado com base nas normas da ABNT, este modelo contém a formatação essencial para apresentação de trabalhos acadêmicos, contempladas em  cinco partes:

O texto pode ser preparado usando LaTeX (ou TeX). Por favor, procure seguir as instruções para que as monografias do DCC possuam uma aparência uniforme.

\section{Figuras}
A impressão de monografias normalmente feita em tons de branco e preto. Portanto, evite fazer uso de fotografias coloridas, a menos que, quando transformadas em tons de cinza seus detalhes continuem visíveis.

As figuras devem ser integradas no texto, centralizadas de acordo com as margens. Para testar a visibilidade dos detalhes de suas figuras, por favor, faça a geração de um arquivo imagem de impressão (postscript) e observem se todos os detalhes estão perfeitamente visíveis e os textos legíveis.  As figuras devem ser numeradas e todas devem ter uma legenda explicativa. 

Tenha especial cuidado com figuras feitas diretamente com as ferramentas MSOffice. Se a figura ocupar uma página completa, certifique-se que esta não ultrapasse as margens. Evite colocar figuras e tabelas no formato paisagem, vide a figura ~\ref{fig:logoufjf}.

\begin{figure}[ht]
 \begin{center}
   \includegraphics{./figs/logoInstituicao.png} % logo-ufjf1.png: 196x111 pixel, 72dpi, 6.91x3.92 cm, bb=0 0 196 111
  \caption{Logotipo da Universidade Federal de Juiz de Fora}
  \label{fig:logoufjf}
 \end{center}
\end{figure}

\section{Fórmulas e equações}
Equações e Fórmulas devem ser colocadas em uma nova linha, centralizadas e numeradas consecutivamente para fins de referência, como pode ser observado na equação ~\ref{eq:exemplo}.
\begin{equation}
\int_{0}^{\infty}f(x)dx = x - \frac{x^3}{3!} + \frac{x^5}{5!} + \cdots = sin(x)
\label{eq:exemplo}
\end{equation}

\section{Algoritmos} 
As listagens de código de programas não são consideradas figuras, de modo que não necessitam ter legenda. Listagens de código geralmente trechos de código retirados de programa, como códigos C, Java ou XML. Para fins de referência, as linhas do código podem ser numeradas.

Por exemplo, o código a seguir mostra uma classe Java, onde a linha 6 inicia um comando que se estende por diversas linhas. 


\lstset{tabsize=5,language=C,showstringspaces=false,basicstyle=\ttfamily\small,keywordstyle=\bf,breaklines=true}
\begin{singlespacing}
\begin{lstlisting}[frame=single,framexrightmargin=1pt,numbers=left]
import java.util.Random;
class Aleatorios {
  public static void main (String[] args) {
     Random qq=new Random();
     for (int k=1;k<10;k++)
        System.out.println(qq.nextInt(100) + "\n" + Math.random());
    }
}
\end{lstlisting}
\end{singlespacing}

 Algoritmos geralmente são apresentados como pseudo-códigos, os quais possuem uma formatação formal conhecida dos livros de computação. Diferentemente das listagens, os algoritmos costumam possuir legendas, como no algoritmo ~\ref{alg:exemplo} abaixo.

\begin{algorithm}
\label{alg:exemplo}
 \caption{Ler número e imprimir se é par ou não.}

     \Entrada{número, ($numero$).}
     \Saida{Se o número é par ou não}
     \Inicio{
         \textbf{ler} $numero$\;
         \eSe {$numero \% 2 = 0$} {
             \textbf{imprimir} $numero$, " par"\;
         } {
         \textbf{imprimir} $numero$, " impar"\;
         }
        }
\end{algorithm}


\section{Tabelas}

Tabelas necessitam de legenda superior, conforme mostrado na tabela~\ref{tab:modelos}.

\begin{table}[ht]
   \centering
   \caption{Porcentagem de modelos por marca}
   \label{tab:modelos}
   \begin{tabular}{| c | c |}
      \hline 
      Marca & Porcentagem \\
      \hline \hline 
      XPTO & 60\% \\
      \hline
      ZWY & 10\% \\
      \hline
      AWK & 10\% \\
      \hline
      HKL & 10\% \\
      \hline
      TPOI & 5\% \\
      \hline
      SSO & 5\% \\
      \hline
   \end{tabular} 
\end{table}

\section{Notas de rodapé}
As notas de rodapé são usadas ao longo do texto para esclarecimentos rápidos sobre algum conceito ou para referência a algum endereço da web, como, por exemplo, na seguinte frase: "A W3C\footnote{http://www.w3.org} é o consórcio regulador da Web".



\section{Citações e referências}
Ao longo do texto, as citações são feitas através do formato (AUTOR, ANO). Ao final, a lista de referências deve ter o nome de Referências Bibliográficas, sem numeração de seção. não colocar quebra de página antes. 

Para inserir referências no documento Latex, se utilize do pacote \textsc{abntex2cite} e uso os comandos \textsc{cite} e \textsc{citeonline}. Com o comando \textsc{cite} as referências ficam como \cite{stojanovic2002} ou \cite{souza2010book, souza2008ismicka} em caso de referências consecutivas. Com o comando \textsc{citeonline} as referências ficam no formato de citação usadas para setenças que citam o autor: "Segundo \citeonline{zhang2007}...".


\section{Comentários}
Durante o trabalho de confecção da monografia, o aluno terá que enviar várias versões para o orientador e este retornaré as versões com comentários sobre o texto. Para facilitar esse trabalho de revisão do aluno e orientador, pode-se incluir no texto comentários do pacote \textsc{todonotes}. Os comentários podem ser dispostos \textsc{na margem} \comment{Este é um comentário na margem da página} ou ser do tipo \textsc{inline}.\comment[inline]{Este é um comentário inline}

Caso você vá inserir uma figura no texto mas ainda não o fez, você pode informar isso ao seu orientador como no exemplo abaixo:
\missingfigure{A figura ainda será criada e inserida aqui.}

Com esse pacote \textsc{todonotes} você pode também inserir em um local da sua monografia um índice de todos os comentários que o seu orientador fez, para que você se lembre do que ainda tem que ser corrigido na monografia. Veja abaixo:

\begin{singlespacing}\listoftodos\end{singlespacing}


\comment[color=green!40]{Lembre-se: todos os comentários e a lista de tarefas \textbf{devem} ser retirados do texto na versão que será enviada para a banca.}


\chapter{Texto de exemplo}
\section{A noção de função}
\indent\indent Iremos trabalhar com a noção intuitiva de função.
Uma definição formal de função, na qual se faz uso da linguagem de
conjuntos e produtos cartesiano, será dada no Anexo I.

Uma função envolve um conjunto $A$, chamado de \textsc{domínio},
um conjunto $B$ chamado de \textsc{contradomínio} e uma regra
denotada por $f:A \rightarrow B$, que nos diz como associar a cada
$a \in A$, um único $f(a)=b \in B$, chamado de \textsc{valor de}
$f$ \textsc{no ponto} $a$ ou \textsc{imagem de} $a$ \textsc{pela
função} $f$.
\begin{example}\label{R2emR}
Seja $A=\R^2$, ou seja, os conjunto de todos os pares ordenados
$(x,y)$ tais que $x,y \in \R$ e seja $B=\R$ o conjunto dos números
reais. Então $f:A \rightarrow B$, definida para cada par $(x,y)$
por $f(x,y)=x$ é uma função, uma vez que, para cada par $(x,y)$,
corresponde um único $x \in \R$.
\end{example}
\begin{example}
Se tomarmos $A=\R$ e $B=\R^2$ a regra $g:A \rightarrow B$,
definida para cada $x \in \R$ por $g(x)=(x,y)$ onde $y \in \R$,
não é uma função pois, para cada $x \in \R$, existem infinitos
pares ordenados $(x,y) \in \R^2$.
\end{example}

\begin{definition}
Duas funções $f:A \rightarrow B$ e $g:C \rightarrow D$ são iguais
se as seguintes condições são satisfeitas:
\begin{enumerate}
    \item $A=C$ e $B=D$;
    \item para cada $a \in A$, $f(a)=g(a)$.
\end{enumerate}
\end{definition}
\begin{definition}
Dada uma função $f:A \rightarrow B$, o subconjunto de $B$ formado
pelos elementos $b=f(a)$, com $a \in A$, é chamado de
\textsc{imagem} de $A$ por $f$, ou \textsc{imagem} de $f:A
\rightarrow B$.
\end{definition}
Usaremos $Im(f)$ ou $f(A)$ para denotar a imagem de $f:A
\rightarrow B$. Portanto, temos
$$
f(A)=\{b=f(a)\in B\;;\; a \in A\}.
$$

\section{Funções Injetivas, Sobrejetivas e Bijetivas}

\begin{definition}
Seja $f:A \rightarrow B$ uma função. Dizemos que:
\begin{enumerate}
    \item $f$ é \textsc{injetiva} (ou \textsc{um-a-um}, ou \textsc{injetora}, ou uma
    \textsc{injeção}) sempre que $a \neq a'$ em $A$, $f(a)\neq
    f(a')$ em $B$;
    \item $f$ é \textsc{sobrejetiva} (ou \textsc{sobre}, ou \textsc{sobrejetora}, ou uma
    \textsc{sobrejeção}) sempre que $f(A)=B$;
    \item $f$ é  \textsc{bijetiva} (ou \textsc{bijetora}, ou uma \textsc{bijeção}) se $f$
    é injetiva e sobrejetiva.
\end{enumerate}
\end{definition}

Observe que, equivalentemente, podemos dizer que $f:A \rightarrow
B$ é injetiva sempre que $f(a)=f(a')$ implicar em $a=a'$.

\begin{example}
A função do exemplo (\ref{R2emR}) é sobrejetiva. De fato, dado $x'
\in \R$, para qualquer $y \in \R$, temos $f(x',y)=x'$. Observe
também que tal função não é injetora pois, por exemplo
$f(2,1)=f(2,3)$, mas $(2,1)\neq (2,3)$.
\end{example}
\begin{definition}
Dados uma função $f:A \rightarrow B$ e $C\subseteq A$, definimos a
\textsc{restrição de} $f$ \textsc{a} $C$ como sendo a função $g:C
\rightarrow B$, definida por $g(c)=f(c)$, para todo $c \in C$.
Normalmente usa-se a notação $f|_C$ para indicar a restrição de
$f$ a $C$.
\end{definition}
\begin{definition}
Sejam $f:A \rightarrow B$ uma função, $C \subseteq A$ e $D
\subseteq B$. Definimos:
\begin{enumerate}
    \item a \textsc{imagem (direta) de} $C$ por $f$ com sendo o subconjunto
    de $f(A)$ dado por
    $$
     f(C)=\{f(x)\;;\; x \in C \},
    $$
    \item  e a \textsc{imagem inversa de} $D$ por $f$ como sendo o subconjunto
    de $A$ dado por
    $$
    f^{-1}(D)=\{a \in A\;;\; f(a) \in D \}.
    $$
\end{enumerate}
\end{definition}

Note que a imagem inversa $f^{-1}(D)$, de um conjunto $D$ por uma
função $f$ pode ser o conjunto vazio, mesmo que $D$ não seja o
conjunto vazio. Por exemplo, considere a função $f:A \rightarrow
B$, onde $A=B=\R$, definida por $f(x)=|x|$. Então, se $$D=\{x \in
\R\;;\; x<0\},$$ teremos que $f^{-1}(D)=\emptyset$, uma vez que
não existe $x \in \R$ tal que $|x|<0$.

\begin{proposition}\label{Img-direta}
Sejam $f:A \rightarrow B$ uma função e $C, D$ subconjuntos de $A$.
\begin{enumerate}
    \item[(a)] Se $C \subseteq D$, então $f(C)\subseteq f(D)$;
    \item[(b)] $f(C \cup D)=f(C)\cup f(D)$;
    \item[(c)] $f(C \cap D) \subseteq f(C)\cap f(D)$.
\end{enumerate}
\end{proposition}
\begin{proof}\mbox{}
\begin{enumerate}
    \item[(a)] Se $y \in f(C)$, então existe $c \in C$ tal que $y
    = f(c)$. Como $C \subseteq D$, segue que $c \in D$ e portanto
    $y=f(c) \in f(D)$, isto é, $f(C)\subseteq f(D)$.
    \item[(b)] Seja $y \in f(C \cup D)$. Então existe $x \in C \cup
    D$ tal que $y=f(x)$. Logo $x \in C$ ou $x \in D$. Se $x \in
    C$, temos que $y=f(x) \in f(C)$. Mas se $x \in D$, teremos
    $y=f(x) \in f(D)$. Logo $y \in f(C) \cup f(D)$ o implica em
    $f(C \cup D)\subseteq f(C)\cup f(D)$.

    Por outro lado, como $C \subseteq C \cup D$, segue do item (a)
    que $f(C) \subseteq f(C \cup D)$. Da mesma maneira temos que $D
    \subseteq C \cup D$ o que implica que $f(D) \subseteq f(C \cup D)$.
    Portanto $f(C)\cup f(D)\subseteq f(C \cup D)$.
    \item[(c)] Como $C \cap D \subseteq C$, segue que $f(C\cap D)
    \subseteq f(C)$. Analogamente, temos $f(C\cap D)\subseteq
    f(D)$. Assim, podemos concluir que $f(C \cap D) \subseteq
    f(C)\cap f(D)$.
\end{enumerate}
\end{proof}

Devemos observar que em geral, no item \emph{(c)} da proposição
(\ref{Img-direta}), não se pode substituir o sinal de inclusão
pelo sinal de igualdade, ou seja nem sempre vale a inclusão
oposta. Por exemplo, sejam $A=B=\R$ e $f:A \rightarrow B$ definida
por $f(x)=x^2$, $C=\{x \in \R\;;\; -1 \leq x <0\}$ e $D=\{x \in \R
\;;\; 0 <x \leq 1\}$. Então
$$
f(C)=f(D)=\{x \in \R \;;\; 0 <x \leq 1\}=f(C)\cap f(D).
$$
Por outro lado $C \cap D= \emptyset$ o que implica em $f(C \cap
D)=\emptyset$. Portanto $$f(C)\cap f(D) \nsubseteq f(C \cap D).$$

\begin{proposition}\label{Img-inversa}
Sejam $f:A \rightarrow B$ uma função e $E, F$ subconjuntos de $B$.
\begin{enumerate}
    \item[(a)] Se $E \subseteq F$, então $f^{-1}(E)\subseteq f^{-1}(F)$;
    \item[(b)] $f^{-1}(E \cup F)=f^{-1}(E)\cup f^{-1}(F)$;
    \item[(c)] $f^{-1}(E \cap F)=f^{-1}(E)\cap f^{-1}(F)$.
\end{enumerate}
\end{proposition}
\begin{proof}\mbox{}
\begin{enumerate}
    \item[(a)] Se $a \in f^{-1}(E)$, então $f(a) \in E$.
    Como $E \subseteq F$, segue que $f(a) \in F$, isto é,
    $a \in f^{-1}(F)$. Portanto $f^{-1}(E)\subseteq f^{-1}(F)$.
    \item[(b)] Como $E \subseteq E \cup F$ e $F \subseteq E \cup
    F$, segue que $f^{-1}(E) \cup f^{-1}(F) \subseteq f^{-1}(E \cup
    F)$.

    Por outro lado, se $a \in f^{-1}(E \cup F)$, então $f(a) \in E
    \cup F$, ou seja, $f(a) \in E$ ou $f(a) \in F$ e isto implica em
    $a \in f^{-1}(E)$ ou $a \in f^{-1}(F)$. Logo $a \in
    f^{-1}(E)\cup f^{-1}(F)$. Portanto $f^{-1}(E \cup F)
    \subseteq f^{-1}(E)\cup f^{-1}(F)$.
    \item[(c)] Como $E \cap F \subseteq E$ e $E \cap F \subseteq
    F$, temos que $f^{-1}(E \cap F)\subseteq f^{-1}(E)\cap
    f^{-1}(F)$.

    Reciprocamente, se $a \in f^{-1}(E) \cap f^{-1}(F)$, então
    $f(a) \in E$ e $f(a) \in F$. Portanto, $f(a) \in (E \cap
    F)$, ou seja, $a \in f^{-1}(E \cap F)$. Logo $f^{-1}(E)
    \cap f^{-1}(F) \subseteq f^{-1}(E \cap F)$.
\end{enumerate}
\end{proof}

\section{Composição de Funções}
\begin{definition}
Sejam $f:A \rightarrow B$ e $g:B \rightarrow C$ duas funções. A
composição $g \circ f$ é a função de $A$ em $C$, definida por $(g
\circ f)(a)=g(f(a))$.
\end{definition}
\begin{example}\label{log-sen}
Considere as funções $\log :\R_+ \rightarrow \R$ e $\mbox{sen}:\R
\rightarrow [0,1]$, onde $\R_+$ é o conjunto dos números reais não
negativos. Então $(\mbox{sen} \circ \log):\R_+ \rightarrow [0,1]$
é a função que a cada $x \in \R_+$ associa o valor
$\mbox{sen}(\log(x))$. Por outro lado, $(\log \circ \mbox{sen})$
nem sempre está definida (não existe $\log( \mbox{sen}(x))$,
quando $x= \frac{3\pi}{2}$). Isso mostra que, em geral $f \circ g
\neq g \circ f$, isto é, a composição de funções é não-comutativa.
\end{example}

\begin{proposition}
Sejam $f:A\rightarrow B$, $g:B \rightarrow C$ e $h:C\leftarrow D$
funções. Então,
$$
h\circ(g \circ f)=(h \circ g)\circ f:A \rightarrow D.
$$
\end{proposition}
\begin{proof}
Basta provar que para cada $a \in A$, $h((g \circ f)(a))=(h \circ
g)(f(a))$. Mas,
$$
h((g \circ f)(a))=h(g(f(a)))=(h \circ g)(f(a)).
$$
\end{proof}
\begin{theorem}
Se $f:A \rightarrow B$ e $g:B \rightarrow C$ são ambas injetoras
(sobrejetora), então $g \circ f$ também é injetora (sobrejetora).
\end{theorem}
\begin{proof}
Vamos mostrar inicialmente que, se $f:A \rightarrow B$ e $g:B
\rightarrow C$ são ambas injetoras, então $g \circ f$ também o é.
Sejam $a_1, a_2 \in A$ tais que $(g \circ f)(a_1)=(g \circ
f)(a_2)$. Então, $g(f(a_1))=g(f(a_2))$. Como $g$ é injetora, segue
que $f(a_1)=f(a_2)$. Mas como $f$ também é injetora, temos
$a_1=a_2$, o que prova que $g \circ f$ é injetora.

Vamos mostrar agora que, se $f:A \rightarrow B$ e $g:B \rightarrow
C$ são ambas sobrejetoras, então $g \circ f$ também o é. Dado $c
\in C$, existe $b \in B$ tal que $g(b)=c$, uma vez que $g$ é
sobrejetora. Por outro lado, como $f$ também é sobre, existe $a \in A$
tal que $f(a)=b$. Logo,
$$
(g \circ f)(a)=g(f(a))=g(b)=c.
$$
Portanto, $g \circ f$ é sobrejetora.
\end{proof}
\begin{corollary}\label{comp-bi}
Se $f:A \rightarrow B$ e $g:B \rightarrow C$ são ambas bijetoras,
então $g \circ f$ também é bijetora.
\end{corollary}

Dado um conjunto $A$, iremos denotar por $\id_A$ a função
identidade de $A$, ou seja, $\id_A:A \rightarrow A$ é a função
definida por $\id_A(a)=a$.

\begin{theorem}\label{inversa-func}
Se $f:A \rightarrow B$ é uma função bijetora, então existe uma
única função também bijetora, $g:B \rightarrow A$, tal que $g
\circ f = \id_A$ e $f \circ g = \id_B$. A função $g$ é chamada de
inversa de $f$ e é geralmente, denotada por $f^{-1}$.
\end{theorem}
\begin{proof}
Dado $b \in B$, como $f$ é bijetora existe $a \in A$ tal que
$f(a)=b$ e se $a' \in A$ tal que $f(a')=b$, então $a'=a$, ou seja,
dado $b \in B$, existe um único $a \in A$, satisfazendo $f(a)=b$.
Denotando tal $a$ por $g(b)$, definimos uma função $g:B
\rightarrow A$. Vamos mostrar que tal função satisfaz as
propriedades do teorema.

Temos que $(f \circ g)(b)=f(g(b))=f(a)=b$, ou seja $f \circ
g=\id_B$. Por outro lado $(g \circ f)(a)=g(f(a))=g(b)=a$, ou seja,
$g \circ f=\id_A$.

Vamos provar agora que $g$ também é bijetora. Segue da definição
de $g$ que a mesma é sobrejetora e portanto, basta mostrarmos que
a mesma e injetora. Sejam $b_1,b_2 \in B$ tais que
$g(b_1)=g(b_2)$. Então, como $f \circ g=\id_B$, temos
$$
b_1=\id_B(b_1)=(f \circ g)(b_1)=f(g(b_1))=f(g(b_2))=(f \circ
g)(b_2)=\id_B(b_2)=b_2.
$$
o que mostra que $g$ é injetora.

Suponha agora que existe $h:B \rightarrow A$ satisfazendo $h \circ
f = \id_A$ e $f \circ h = \id_B$. Então,
$$
h=h \circ \id_B= h \circ (f \circ g)= (h \circ f) \circ g)=\id_A
\circ g= g.
$$
Logo $g$ é única.
\end{proof}
\begin{remark}
Note que se $g=f^{-1}$, então $f=g^{-1}$.
\end{remark}

Se uma função $f:A \rightarrow B$ admite uma inversa, dizemos que
a mesma é \textsc{inversível}. Portanto, pelo teorema
(\ref{inversa-func}), toda função bijetora é inversével. O próximo
resultado mostra que a recíproca desta afirmação também é
verdadeira.

\begin{proposition}\label{comp-bije}
Se $f:A \rightarrow B$ e $g:B \rightarrow A$ são funções que
satisfazem $g \circ f = \id_A$ e $f \circ g = \id_B$, então ambas
são bijetoras.
\end{proposition}
\begin{proof}
Vamos provar que $g$ é bijetora (a prova para $f$ é análoga).
Sejam $b_1, b_2 \in B$ tais que $g(b_1)=g(b_2)$. Então
$f(g(b_1))=f(g(b_2))$. Como $f \circ g = \id_B$, temos que
$\id_B(b_1)=\id_B(b_2)$, ou seja, $b_1=b_2$, provando que $g$ é
injetora.

Dado $a \in A$, temos
$$
a=\id_A(a)=(g \circ f)(a)=g(f(a)).
$$
Assim, tomando $b=f(a)$ teremos $g(b)=a$ e isto prova que $g$ é
sobrejetora.
\end{proof}



%%%%%%%%%%%%%%%%%%%%%%%%%%%%%%%%%%%%%%%
\singlespacing
\bibliographystyle{abntex2-alf}
\bibliography{referencias}

%%%%%%%%%%%%%%%%%%%%%%%%%%%%%%%%%%%%%%%%%%%%%%%%

\end{document}
